\documentclass{homework}
\usepackage{amsmath}

\title
{
CSE4088 Introduction to Machine Learning 
Homework 3
}

\date{}
\author{Berk Kırtay - 150118043}
\begin{document}
\maketitle

\section*{Gradient Descent}
For this section, Gradient_Descent.py is implemented. Script can be run to observe results. Plotting is included.\\
Results after running Gradient_Descent.py:\\
\includegraphics[scale=1]{gradient_descent.png}

\subsection*{4.}
The partial derivative of E(u,v) with respect to \(\frac{\partial E}{\partial u}\) is

\[ \frac{\partial E}{\partial u} = 2*(u*e^v - 2*v*e^{-u}) * (e^v + 2*v*e^{-u})\]
\subsection*{5. and 6.}
In this section we apply gradient descent algorithm to the nonlinear error space function by 
calculating the derivatives of squared residuals until error becomes less than \(10^{}-14\).
Derivatives to apply gradient descent:
\[ \frac{\partial E}{\partial u} = 2*(u*e^v - 2*v*e^{-u}) * (e^v + 2*v*e^{-u})\]
\[ \frac{\partial E}{\partial v} = 2*(u*e^v - 2*v*e^{-u}) * (u*e^v - 2*e^{-u})\]

Number of iterations until error becomes 1.2086833944220747e-15
is 10.\\
The latest (u,v) pair is (0.04473629039778207, 0.023958714099141746)\\

Plotting of the (u,v) pair on every iteration:\\
\includegraphics[scale=1]{gradient_descent1.png}
The answers are 5: D, and 6: E

\subsection*{7.}
The error does not converge and stops at 0.13981379199615315 which is close to \(10^-1\)

The latest (u,v) pair is (6.29707589930517, -2.852306954077811))\\

Plotting of the (u,v) pair on every iteration:\\
\includegraphics[scale=1]{gradient_descent2.png}
The answer is A.

\section*{Logistic Regression}
For this section, Logistic\_Regression.py is implemented. Script can be run to observe results. Plotting is included.\\
Results after running Logistic\_Regression.py:\\
\includegraphics[scale=1]{logistic_regression.png}\\

The used formulas for the logistic regression algorithm are:\\
\[\text{Fixed step size: } w(1) = w(0) + n\nabla E_{in} \]
\[\text{v is calculated by stochastic gradient descent: } \nabla E_{in} = -\frac{1}{N} \sum_{n=1}^N \frac{y_n*x_n}{ln(1 + e^{y_n*w^T(t)*x_n}})  \]
\[\text{cross entropy error: }E_{out}(w) = \frac{1}{N} \sum_{n=1}^N ln(1 + e^{-y_n*w^T*x_n})\]

More details can be found in the source-code.

Predicted line on 100 points by logistic regression algorithm:\\
\includegraphics[scale=1]{logistic_regression1.png}

\subsection*{8.} 
I used the cross entropy error formula to calculate \(E_out\).\\

Average Eout after 100 runs: 0.09444365341119473.\\
The answer is D.
\subsection*{9.} 
Average epochs after 100 runs: 338.58.\\
The answer is A.

\section*{Regularization with weight decay}
For this section, Linear\_Regression.py is implemented. Script can be run to observe results. Some data scattering is included.\\
Results after running Linear\_Regression.py:\\
\includegraphics[scale=1]{linear_regression.png}\\

\subsection*{2.} 
Actual in-sample data: \\
\includegraphics[scale=0.8]{q21.png}\\
Predicted in-sample data: \\
\includegraphics[scale=0.8]{q22.png}\\

Actual out-of-sample data: \\
\includegraphics[scale=0.8]{q23.png}\\
Predicted out-of-sample data: \\
\includegraphics[scale=0.8]{q24.png}\\

Ein: 0.02857142857142857, Eout: 0.084 \\
The answer is A.
\subsection*{3.} 
0.02857142857142857, Eout: 0.08 \\
The answer is D.
\subsection*{4.} 
Q4: Ein: 0.37142857142857144, Eout: 0.436 \\
The answer is E.
\subsection*{5.} 
k = -1 \\
The answer is D.
\subsection*{6.} 
Eout: 0.056 \\
The answer is B.

\section*{Neural Networks}
For this section, Neural\_Network.py is implemented using the formulas below. Script can be run to observe results.\\
\textbf{Neural Network operators:}\\
\[
w_{ij}^{(l)} =
\left\{
	\begin{array}{ll}
		  1 \leq l \leq L \text{ ~~~~~~~~Layers}  \\
		  0 \leq i \leq d^{(l - 1)} \text{ ~~~Inputs}  \\
            1 \leq j \leq d^{(l)} \text{ ~~~~~Outputs}  \\
	\end{array}
\right\}
\]
\textbf{Activation function: }
\[\frac{(e^s - e^{-s})}{(e^s + e^{-s})} \]
\textbf{Forward propagation: }
\[x_j^{(l)} = \theta(s_j^{(l)}) = \theta(\sum_{i = 0}^{d^{(l - 1)}} w_{ij}^{(l)} * x_i^{(l - 1)})\]
\textbf{Backward propagation: }\\
\[\text{For final layer: } \delta_1^{(L)} = 2*(x_1^{(L)} - y_n)*(1 - \theta^2(s_1^{(L)}))\]

\[\text{For other layers: } \delta_i^{(l - 1)} = (1 - (x_i^{(l - 1)})^2) * \sum_{j = 1}^{d^{(l)}} w_{ij}^{(l)}* \delta_j^{(l)}\]

\[\text{Updating perceptron weights: } w_{ij}^{(l)} \leftarrow w_{ij}^{(l)} - n*x_i^{(l - 1)} * \delta_j^{(l)}\] \\

Results after running Neural\_Network.py:\\
\includegraphics[scale=1]{neural_network.png}\\

\subsection*{8.} 
Total number of operations is: 28. More details can be observed from run_q8() function in the source-code \\
The answer is A.
\subsection*{9.} 
By setting 36 hidden layers with 1 weight we count all the weights in the neural network
as 46. More details can be observed from run_q9() function in the source-code \\
The answer is A.
\subsection*{10.} 
By setting 2 hidden layers with weights of 18, 18 we count all the weights in the neural network as 522. More details can be observed from run_q10() function in the source-code.\\ 
The answer is E.
\end{document}