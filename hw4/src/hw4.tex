\documentclass{homework}
\usepackage{amsmath}

\title
{
CSE4088 Introduction to Machine Learning 
Homework 4
}

\date{}
\author{Berk Kırtay - 150118043}
\begin{document}
\maketitle

\section*{Polynomial Kernels}
For all the sections, hw4.py is implemented. More details and comments can be found in the source code. The functions q2(), q3(), q4() and q5_q6() must be run in the hw4.py to acquire the answers to the questions of this section.\\
Results after running the mentioned functions:\\
\includegraphics[scale=0.7]{first_section.png}

\subsection*{2.}
The classifier 0 has the highest Ein among others with Ein: 10.588396653408317%.
So, the answer is A.

\subsection*{3.}
The classifier 1 has the lowest Ein among others with Ein: 1.4401316691811772%.
So, the answer is A.

\subsection*{4.}
Difference between the number of support vectors of these two classifiers is 1793.
It is the closest to the 1800, so, the answer is C.

\subsection*{5.}
Answer dd must be correct because the maximum C
achieves the lowest Ein and all other values do not strictly
decrease as it is mentioned in the other answers.
The output of this question:\\
\includegraphics[scale=0.7]{q5.png}

So, the answer is D.
\subsection*{6.}
All other choices are incorrect except B. The number of support vectors is
lower at Q=5 for C=0.001.
The output of this question:\\
\includegraphics[scale=0.7]{q6.png}
So, the answer is B.

\section*{Cross Validation}
The functions q7() and q8() must be run in the hw4.py to acquire the answers to the questions of this section.\\
Results after running the mentioned functions:\\
\includegraphics[scale=1]{second_section.png}

\subsection*{7.} 
As we can see from the output, C=0.001 is most selected with 37 times over 100 runs.
So, the answer is B.
\subsection*{8.} 
As we can see from the output, the average value of Ecv over the 100 runs is 0.0049 which is the closest to 0.005.
So, the answer is C

\section*{RBF Kernel}
The function q9()_q10() must be run in the hw4.py to acquire the answers to the questions of this section.\\
Results after running the mentioned function:\\
\includegraphics[scale=1]{third_section.png}

\subsection*{9.} 
1000000=\(10^6\) gives the lowest value of Ein.
So, the answer is E.
\subsection*{10.} 
100 gives the lowest value of Ein.
So, the answer is C.
\end{document}